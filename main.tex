%%%%%%%%%%%%%%%%%%%%%%%%%%%%%%%%%%%%%%%%%%%%%%%%%%%%%%%%%%%%%%%%%%%%%%%%
%    INSTITUTE OF PHYSICS PUBLISHING                                   %
%                                                                      %
%   `Preparing an article for publication in an Institute of Physics   %
%    Publishing journal using LaTeX'                                   %
%                                                                      %
%    LaTeX source code `ioplau2e.tex' used to generate `author         %
%    guidelines', the documentation explaining and demonstrating use   %
%    of the Institute of Physics Publishing LaTeX preprint files       %
%    `iopart.cls, iopart12.clo and iopart10.clo'.                      %
%                                                                      %
%    `ioplau2e.tex' itself uses LaTeX with `iopart.cls'                %
%                                                                      %
%%%%%%%%%%%%%%%%%%%%%%%%%%%%%%%%%%
%
%
% First we have a character check
%
% ! exclamation mark    " double quote  
% # hash                ` opening quote (grave)
% & ampersand           ' closing quote (acute)
% $ dollar              % percent       
% ( open parenthesis    ) close paren.  
% - hyphen              = equals sign
% | vertical bar        ~ tilde         
% @ at sign             _ underscore
% { open curly brace    } close curly   
% [ open square         ] close square bracket
% + plus sign           ; semi-colon    
% * asterisk            : colon
% < open angle bracket  > close angle   
% , comma               . full stop
% ? question mark       / forward slash 
% \ backslash           ^ circumflex
%
% ABCDEFGHIJKLMNOPQRSTUVWXYZ 
% abcdefghijklmnopqrstuvwxyz 
% 1234567890
%
%%%%%%%%%%%%%%%%%%%%%%%%%%%%%%%%%%%%%%%%%%%%%%%%%%%%%%%%%%%%%%%%%%%
%

% Include comments
%At the start of the LA T EX source code please include commented material to identify the journal, author, and (if you are sending a revised version or a resubmission) the reference number that the journal has given to the submission. 

\documentclass[12pt]{iopart}
\newcommand{\gguide}{{\it Preparing graphics for IOP Publishing journals}}
%Uncomment next line if AMS fonts required
%\usepackage{iopams}  
\usepackage{cite}
\usepackage[english]{babel}




\begin{document}
	
\title[Silage Maize Yield estimates using Soil Moisture Climate Projections]{Silage Maize Yield estimates using Soil Moisture Climate Projections}
\author{Michael Peichl$^1$, Stephan Thober$^1$, Andreas Marx$^{1,2}$}

\address{$^1$ Department Computational Hydrosystems, Helmholtz Centre for Environmental Research - UFZ, Permoserstrasse 15, D-04318 Leipzig, Germany}
\address{$^2$ Climate Office for Central Germany, Helmholtz Centre for Environmental Research - UFZ, Permoserstrasse 15, D-04318 Leipzig, Germany}
\ead{michael.peichl@ufz.de}

\begin{abstract}
Here comes the abstract. The abstract should normally be restricted to a single paragraph of around 200 words.



\end{abstract}

\noindent{\it Keywords\/}: silage maize, climate change, Germany (3 - 7 words)

%\keywords{magnetic moment, solar neutrinos, astrophysics}

%Uncomment for PACS numbers title message
%\pacs{00.00, 20.00, 42.10}
% Keywords required only for MST, PB, PMB, PM, JOA, JOB? 
%\vspace{2pc}
%\noindent{\it Keywords}: Article preparation, IOP journals
% Uncomment for Submitted to journal title message
\submitto{\ERL}
% Comment out if separate title page not required
\maketitle
	

\section{Introduction}
Should state clearly the object of the work, its scope and the main advances reported, with brief references to relevant
results by other workers   \cite{Brouwer2012, Schlenker2009}.

\section{Data}
\subsection{Training Data}
\subsection{Climate Data}
\section{Methods}
\section{Results}
\section{Discussion}
% \ack
% \appendix 

\bibliography{Mendeley}{}
\bibliographystyle{plainnat}
	
\end{document}



